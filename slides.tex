\documentclass{beamer}
\usepackage{tikz}
\usetheme{Berkeley}
\usecolortheme{default}

\mode<presentation>

\definecolor{pghyellow}{RGB}{255,191,0}
\definecolor{pghblack}{RGB}{19,23,31}
\definecolor{pghdarkergrey}{RGB}{24,28,37}
\definecolor{pghdarkgrey}{RGB}{26,31,40}

\setbeamercolor*{palette primary}{fg=pghyellow,bg=pghblack}
\setbeamercolor*{palette secondary}{fg=pghyellow,bg=pghblack}
\setbeamercolor*{palette tertiary}{fg=white,bg=pghblack}
\setbeamercolor*{palette quaternary}{fg=white,bg=pghblack}

\setbeamercolor*{sidebar}{fg=pghyellow,bg=pghdarkergrey}

\setbeamercolor*{palette sidebar primary}{fg=pghyellow, bg=pghdarkgrey}
\setbeamercolor*{palette sidebar secondary}{fg=pghyellow}
\setbeamercolor*{palette sidebar tertiary}{fg=pghyellow}
\setbeamercolor*{palette sidebar quaternary}{fg=pghyellow}

\setbeamercolor*{titlelike}{parent=palette primary}
\setbeamercolor{titlelike}{parent=palette primary,fg=pghyellow}
\setbeamercolor{frametitle}{bg=pghblack, fg=white}
\setbeamercolor{frametitle right}{bg=pghblack, fg=white}

\mode
<all>

% <TITLE CONFIG BLOCK>

\title{Furbanism and Transit}
\subtitle{Transit in Pittsburgh, Furries, and what you can do! By Aured, Nick, Kezl and Teg}
\date[Anthrocon 2025]{Anthrocon, July 2025}
\logo{\includegraphics[height=0.8cm]{img/aclogo.png}}
    
% </TITLE CONFIG BLOCK>

\begin{document}

\setbeamercolor{background canvas}{bg=pghdarkgrey}
\setbeamercolor{normal text}{fg=white}
\usebeamercolor[fg]{normal text}
\frame{\titlepage}

\begin{frame}
\frametitle{About us}
\begin{itemize}
    \item Aured: short bio % TODO: Elaborate!
    \item Kezl: short bio % TODO: Elaborate!
    \item Nick: short bio % TODO: Elaborate!
    \item Teg: short bio % TODO: Elaborate!
\end{itemize}
\end{frame}

\begin{frame}
\frametitle{Pittsburgh}
Located in Allegheny County, PA \\
Population: 302,971 (2020 Census) (has steadily declined since its peak in 1950-1975) \\
\end{frame}

\begin{frame}
\frametitle{Transit}
Pittsburgh Regional Transit is the primary transit operator in Allegheny County. \\
Fleet of approximately x vehicles: \\
a bus routes, b rail routes, 2 inclines \\
Infrastructure: 3 busways, HOV lane, light rail network with at-grade and grade separated segments
\end{frame}

\begin{frame}
\frametitle{The T (short for Trolley!)}
\end{frame}

\begin{frame}
\frametitle{West Busway}
\includegraphics[width=\textwidth]{img/westbusway.png}
\end{frame}

\begin{frame}
\frametitle{East Busway}
\begin{itemize}
    \item 9 stations (Penn, Herron, Negley, E. Liberty, Homewood, Wilkinsburg, Hamnett, Roslyn, Swissvale)
    \item 14 routes: 2 are "busway" routes, rest are "flyer" routes 
    \item P1, P3 are busway routes. P7, P10, P12, P13, P16, P17, P67, P68, P69, P71, P76, P78 are flyer routes
\end{itemize}
\end{frame}

\begin{frame}
\includegraphics[width=\textwidth]{img/eastbusway.png}
\end{frame}

\begin{frame}
\frametitle{PRTX/University Line/BRT}
\includegraphics[width=\textwidth]{img/brtmap.png}
\end{frame}

\end{document}